\documentclass[9pt,twocolumn,twoside,]{pnas-new}

%% Some pieces required from the pandoc template
\providecommand{\tightlist}{%
  \setlength{\itemsep}{0pt}\setlength{\parskip}{0pt}}

% Use the lineno option to display guide line numbers if required.
% Note that the use of elements such as single-column equations
% may affect the guide line number alignment.


\usepackage[T1]{fontenc}
\usepackage[utf8]{inputenc}



\templatetype{pnasresearcharticle}  % Choose template

\title{The Educational Achievement and Opportunity Gaps in Mexico}

\author[a]{Ollin Demian Langle Chimal}

  \affil[a]{University of Vermont, Complex Systems and Data Science, Burlington,
Vermont, 05401}


% Please give the surname of the lead author for the running footer
\leadauthor{Langle Chimal}

% Please add here a significance statement to explain the relevance of your work
\significancestatement{The study of education in marginalized areas and the achievement gap
compared to developed areas has been studied greatly but little
understanding for the causes of it other than nutrition and teacher
quality has been made. Here we show that there is a corralation between
the gap in the average achievement in language and mathematics and the
indigenous population implying a structural disadvantage to the students
with mother tongue other than Spanish.}


\authorcontributions{}



\correspondingauthor{\textsuperscript{} }

% Keywords are not mandatory, but authors are strongly encouraged to provide them. If provided, please include two to five keywords, separated by the pipe symbol, e.g:
 \keywords{  education |  opportunity gap |  indigenous language |  causation  } 

\begin{abstract}
Mexico is one of the worst performing countries in the PISA education
evaluation program even though the history to fix the lag across the
country has led to the implementation of novelty programs such as
PROGRESA which introduced the conditional economic subsidies for
development. The access to quality education in Mexico, as well as
around the World, is greatly correlated to wealth and it's been long
studied as one causation of the poverty trap. Using results from the
mexican educational evaluation program ``PLANEA'' we analize the drivers
for the achievement level distribution in mathematics and language and
their difference to eachother, finding that municipalities with high
marginalization degree tend to perform worst in language while low
marginalization can be associated to a poorer performance in
mathematics. Marginalization is highly correlated to the percentage of
indigenous population which suggest that the disparity between
achievement levels in language and math in that group could be due the
testing and teaching in other than the mother tongue rather than a
cognitive implication.
\end{abstract}

\dates{This manuscript was compiled on \today}
\doi{\url{www.pnas.org/cgi/doi/10.1073/pnas.XXXXXXXXXX}}

\begin{document}

% Optional adjustment to line up main text (after abstract) of first page with line numbers, when using both lineno and twocolumn options.
% You should only change this length when you've finalised the article contents.
\verticaladjustment{-2pt}

\maketitle
\thispagestyle{firststyle}
\ifthenelse{\boolean{shortarticle}}{\ifthenelse{\boolean{singlecolumn}}{\abscontentformatted}{\abscontent}}{}

% If your first paragraph (i.e. with the \dropcap) contains a list environment (quote, quotation, theorem, definition, enumerate, itemize...), the line after the list may have some extra indentation. If this is the case, add \parshape=0 to the end of the list environment.

\acknow{I want to recognize the work of every community teacher whose labor for
clossening the gap and work for children's wellness and rights has
inspired this work. Also to all of those who collect and share
information.}

The National Plan for the Learning Evaluation (PLANEA for it's Spanish
initials) is part of the Ministry of Public Education (SEP) efforts to
evaluate the acedemic performance accross Mexico with the aim for
targeted public policies. PLANEA performs several types of test to keep
track of academic achievements which has been shown importance (1), the
three modalities are a national evaluation, a scholar center evaluation
and a diagnostic evaluation, the first two evaluate the academic
achievements at the end of both levels of obligatory education, primary
(2) and secondary. Unlike it's predecessors such as ENLACE, it evaluates
nearly the total amount of schools but only for the diagnostic
evaluation on the equivalent of 6 and 9 years of consecutive education,
those are called obligatory as the government has the consitutional
obligation to provide it to every person in the Country. The track of
academic scores has been a major interest of the mexican government
since the 1990s where the program PROGRESA (3) and later OPORTUNIDADES
(4) (5) which were the first conditional cash transfer programs in the
World to be established and they wanted to evaluate their impact in the
walfare (6). The focus of this program was to mantain school attainment
(7), which is now consider not the best variable to track as there could
be presence but no education happening in a school (8). The biggest
difference between Mexico's approach and the rest of the region is that
Mexico focuses in public education while other countries like Colombia
have voucher programs like PACES for children to be educated in private
schools (9) (10). The latest test was performed on 2019 accounting for
32,390 schools for a total of 1,015,597 students evaluated on the
language part and 1,011,926 on mathematics. This over 1,992 of the 2,457
municipalities in Mexico at that time. The reslts for each school are
summarized by the percentage of students in each of the 4 levels of
achievement for both areas.

\begin{table}[H]
  \centering
  \begin{tabular}{|c|c|}
  \hline
  Level & Equivalence \\ \hline
  IV                       & Outstanding                       \\ \hline
  III                       & Satisfactory                       \\ \hline
  II                       & Sufficient                       \\ \hline
  I                       & Insufficient                       \\ \hline
  \end{tabular}
\caption{Achievement levels in the PLANEA test}
\label{table:levels}
\end{table}

In order to understand how the context affects education other data
sources are necessary. There is an extensive literature that reviews the
socioeconomic effects on schooling (1). The National Population Council
(CONAPO) runs an evaluation of marginalization by municipality every 5
years which is a multidemensional metric that takes four category needs
into account; education, dwelling, population distribution and monetary
income, the educational part takes into account the percentage of
illiterate over the age of 15 people and the percentage of people over
15 that didn't the 6th grade of education. This multidimensional metric
stratifies the municipalities into 5 groups; very low, low, medium, high
and very high.

\begin{table}[H]
     \centering
     \begin{tabular}{|c|c|c|}
     \hline
     Marginalization & Number of municipalities & Number of schools\\ \hline
     Very Low & 345 & 13153\\ \hline
     Low & 498 & 6199\\ \hline
     Medium & 514 & 5708\\ \hline
     High & 817 & 6777\\ \hline
     Very High & 283 & 2372\\ \hline
     \end{tabular}
     \caption{Number of municipalities and schools by marginalization level. There were 5 schools in municipalities without marginalization information.}
     \label{tab:marginalization}
\end{table}

The number of municipalities with high marginalization degree exceeds
the ones with lower values of it but the number of schools in those are
more as we can see from table\ref{tab:marginalization}.

The impact of the circumstances on the achievement levels can be seen in
figures \ref{fig:achlevels} and \ref{fig:achlevelsmar} which are very
similar but let us see different trends in the data. Figure
\ref{fig:achlevels} shows the effect of the school being in a certain
level of marginalization on distribution of the student's percentage
given the type of school, on the other hand figure
\ref{fig:achlevelsmar} depicts the effect of the school type on the
levels of marginalization which has been discussed in (11) and
conditioned to gender in (12). On the first one something very important
emerges on the language test, the median over type of schools is
somewhat preserved over all marginalization levels but for the very high
one and this effect is not equally extreme for the mathematics test, is
important to analyze this with statistical rigor. The second figure
shows an equally interesting but unsurprising result, regardless of the
marginalization level private schools' students perform better on the
test.

\begin{figure*}
    \centering
    \includegraphics[width=\textwidth,height=10cm]{figs/achievement_levels.png}
    \caption{Distribution of student percentage of a school in each achievement level grouped by marginalization index and school type.}
    \label{fig:achlevels}
\end{figure*}

\begin{figure*}
    \centering
    \includegraphics[width=\textwidth,height=10cm]{figs/achievement_levels_by_marginalization.png}
    \caption{Distribution of student percentage of a school in each achievement level grouped by school type and marginalization index.}
    \label{fig:achlevelsmar}
\end{figure*}

Because of the aggregated format in which the data is released it is not
possible to have a unique target variable, instead we have the
distributions of the percentage of students by school that falls inside
a certain level group, thus we have 4 variables of interest.

\begin{figure}[H]
\centering
\includegraphics[width=\columnwidth]{figs/stacked_math.png}
\caption{Choropleth map for the mean percentage of students by achievement levels of the mathematics part of the PLANEA test.}
\label{fig:stackedmath}
\end{figure}

The figure \ref{fig:stackedmath} shows the geographical distribution of
the mean percentage of students by municipality with scores equivalent
to each achievement level. The number of students with insufficient
grades greatly exceeds those on the other mark groups. This map also let
us see that most of the municipalities without data are in Oaxaca where
the indigenous population is high. The zone with higher mean values of
achievement is the north-west region, specifically Sonora and Sinaloa.

\begin{figure}[H]
\centering
\includegraphics[width=\columnwidth]{figs/stacked_language.png}
\caption{Choropleth map for the mean percentage of students by achievement levels of the language part of the PLANEA test.}
\label{fig:stackedlan}
\end{figure}

On the other hand figure \ref{fig:stackedlan} shows that the
distributions for the language part of the test are quite different from
those for mathematics. This subject-matter shows that even though the
distribution is still skewed towards the lower levels, the difference
between the I and II groups is dimnished compared to the one in figure
\ref{fig:stackedmath} with 6.1\% of the students compared to 8.5\% in
mathematics, in the case of the level I groups the mean values are
37.8\% in language and 56.8\% in mathematics. This means that less
students fail badly in the language tasks but also less students excel.

It's important to note that as both test weren't performed one after the
other, the number of students who took them is not the same, we have the
information of 1,012,267 students for the mathematical part and
1,016,087 in the language counterpart. The data contains a column about
if the percentage of students present in the test are a representative
sample of the school, but the documentation neglects to explain the
methodology and by manual inspection we found that is not only based on
the percentage of students present as there are different cut-offs for
different schools and test part, then, as we don't have the information
to asume this as valide we kept those school where the percentage in
both tests were more than the 50\%. Then the number of students
evaluated in language was reduced to 1,008,233 and to 1,004,766. Still
it contains above a million records for both. Now the number of schools
kept is 31,554 in 1,988 municipalities.

\hypertarget{achievement-difference-between-language-and-mathematics}{%
\section*{Achievement difference between language and
mathematics}\label{achievement-difference-between-language-and-mathematics}}
\addcontentsline{toc}{section}{Achievement difference between language
and mathematics}

When evaluating the academic performance by regions we usually see
one-to-one comparisons between them for the same subject-matters but the
difference of achivements on them could give us valuable information
about the structure of the educational planning.

\begin{figure}[H]
\subfigure{\includegraphics[width=\columnwidth]{figs/ridges_marg_I.png}
\label{fig:ridgeI}}
\quad
\subfigure{\includegraphics[width=\columnwidth]{figs/ridges_marg_IV.png}
\label{fig:ridgeIV}}
\caption{Distribution density in log10 scale of the percentage of students by municipality in the lowest and highest achievement level.}
\label{fig:ridgeMarg}
\end{figure}

The PLANEA program tests for two different areas of cognitive
development and in a completely equal disadvantage we should expect to
see the same correlation between same level of achievements between the
both of them but figure \ref{fig:ridgeMarg} shows that the log
distribution of the percentage of students in the level one groups are
different for the very low marginalization level (p-value: 2.58e-8, two
sided F-test) while they are the same for very the very high one
(p-value: 6.70e-1, two sided F-test). As the hypothesis is that the gap
is the same in every municipality, i.e.~we have the same disadvantage
for both test parts, we used a F-test to compare the variances of the
two samples and see how good of a fit is one for the other. For the
level IV group we conclude that for the very low marginalization
municipalities the log distributions are the same (p-value: 6.44e-1, two
sided F-test) while for the very high marginalization it's inconclusive
depending on the significance threshold (p-value: 7.73e-2, two sided
F-test). Visual inspection of figure \ref{fig:ridgeMarg} shows that
there is indeed a difference. In particular for the language part of the
test the municipalities with a very low marginalization level more
students' percentage have an insufficient mathematical performance in
the test than they do on the language areas but that difference vanishes
on the very marginalized ones where they perform as poorly. On the level
IV this behaviour is reversed, municipalities with low marginalization
that do well in one do it too on the other but as marginalization
increases less percentage of students with outstanding performance in
mathematics excels in language too.

The previous results suggests that there is a non-cognitive variable
associated with the gap between the parts of the test. The overall
performance difference is usually explained by the inaccessibility to
schools or quality education (13) (14), working children, parent
meritocracy (15) (16) (17), lack of nutritious food \textbf{???} (18),
absent educators (8), lack of estimulation outside the academic
environment (19) (20), etc., but all of these account for the same in
both parts of the test and there should be an enviromental variable that
explains this pattern of children in unfavored conditions which can
affect their later entrance to a school with standardize admision test
and poor integration (21) (22).

\begin{figure}[H]
\centering
\includegraphics[width=\columnwidth]{figs/indi_vs_marg.png}
\caption{Distribution of indigenous population by marginalization level.}
\label{fig:indiMarg}
\end{figure}

Being Mexico the large multicultural country that is, it's important to
understand the drivers of this phenomena as the integration of all the
nations inside the country is not yet achieved and the unlike other
developing countries such as Malasya (23), inequality between groups is
very large. Using the intercensal survey gathered by the National
Population Council (CONAPO) and the National Comission for the
Development of Indigenous Peoples (CDI) we were able to cross the
information of percentage of people that speaks an indigenous language
and the variables that make up the multidimensional definition of
poverty and marginalization in Mexico which have been studied in the
relationship between schools and inequality (24) (25). Figure
\ref{fig:indiMarg} depicts that the most marginalized areas in Mexico
have a high concentration of indigenous people and therefore a social
explanation of why this lag of language development is happening in
those regions (26) (27) (28) .

\hypertarget{causality-of-the-achievement-levels}{%
\section*{Causality of the achievement
levels}\label{causality-of-the-achievement-levels}}
\addcontentsline{toc}{section}{Causality of the achievement levels}

Understanding the mechanisms by which a phenomena occurs is the the most
important part for program planning or any type of public policy
directed to tackle an specific problem and while descriptive statistics
can yield important findings the use of inferential statistics is
crucial to extract any conclusions about a certain population and
advance into policy treatments (29). Then in order to understand what
specific variables have an impact on the test scores we performed a
causality analysis which is regularely used in education programs (30).
The variables we used were illiteracy, percentage of adult population
without elementary school, percentage of people living under 2 minimum
wages a month, percentage of dirt floors in houses, municipalities with
less than 5 thousand population, percentage of houses without
electricity access, percentage of houses without sewage, percentage of
houses without clean water access, percentage of overcrowding, school
shift, school type and schools by square km in the municipality. After a
feature simple feature selection process we only kept the eight less
correlated ones and use a Hybrid Hybrid Parents \& Children (H2PC)(31)
algorithm to learn the structure of the directed acyclic graphs and then
used the DAG structure to do a regression analysis.

\begin{figure*}
\subfigure{\includegraphics[width=\columnwidth]{figs/dag_I_len.png}
\label{fig:dagIlen}}
\quad
\subfigure{\includegraphics[width=\columnwidth]{figs/dag_IV_len.png}
\label{fig:dagIVlen}}
\quad
\subfigure{\includegraphics[width=\columnwidth]{figs/dag_I_mat.png}
\label{fig:dagImat}}
\quad
\subfigure{\includegraphics[width=\columnwidth]{figs/dag_IV_mat.png}
\label{fig:dagIVmat}}
\caption{DAG structure for causality of the extreme groups (I and IV) for the language (above) and mathematics (below) tasks in the PLANEA test.}
\label{fig:dags}
\end{figure*}

After learning the DAG structure for the language scores we find that
the I and IV groups share most of their parents, the intersection and
their values of the fitted regression analysis for the group I is school
type (-0.063), school shift (0.048), school density (-0.045) and
overcrowding (0.005) with \(\sigma=0.28\). The only non categorical one
with a positive relation is density, thus the more schools per area less
failures in language tasks. The shared regression parameters for the
group IV are school type (0.082), school shift (-0.049), school density
(0.038) and overcrowding (-0.007) with \(\sigma=0.32\). The signed are
reversed with respect to the ones from group I which is expected as they
are the extreme values. It is interesting that no indigenous population
variable were directly linked to them and even on the other extreme of
the leaves.

In the case of the mathematics task both levels share the same parents,
the regression values for the level I are school shift (0.021), school
type (-0.056), overcrowding (0.001) and percentage of indigenous
language speakers (-0.068) on the other hand for the level IV group we
have school shift (-0.033), school type (0.094), overcrowding (-0.003)
and percentage of indigenous language speakers (0.173). Again we have
the parameters' signs shifted but interestingly they also are with
respect of the same achievement level groups for the language task.
Surprisingly the percentage of indigenous language speakers appears as a
parent for both of them when it wasn't for the language ones.

\hypertarget{conclusions}{%
\section*{Conclusions}\label{conclusions}}
\addcontentsline{toc}{section}{Conclusions}

We have shown that there is a difference between the PLANEA test scores
between the language and mathematics part of it for the students in the
most marginalized areas compared to the least ones. This cannot be
explained by common approaches to understand the educative lag as they
account for an overall achievement without acknowledging the different
groups' scores for different subject-matters. We presented that this
difference does exist and is significant and translates into a
disadvantage in language education for the most marginalized areas where
it happen to be more densely populated by indigenous communities. This
communities often times doesn't have Spanish as their first
languanguage. There have been efforts for incorporating the original
languages into public education but as the admission test as well as
this type of evaluations such as PLANEA and PISA are made in Spanish,
the people in those areas have a clear handicap compared to less
marginalized areas. This is in top of the rest of structural skewness in
the access to quality education and nutritious food, not to mention the
problem of working children and the generalized violence in Mexican
rural areas. We can't stress enough that we are not proposing that
education should have a structural change towards colonialist education
but there is a need for better opportunities that can close not only
this but the rest of opportunity gaps, and this research aims to
acknowledge a problematic that should be fixed. We also showed that
school type is causal of the test scores as well as the school density.
We didn't find with this approach a direct parent/child relationship
between the language scores and the indigenous population, further work
on this is needed.

\showmatmethods
\showacknow
\pnasbreak

\hypertarget{refs}{}
\leavevmode\hypertarget{ref-Brunello2007}{}%
1. Brunello G, Checchi D (2007) Does school tracking affect equality of
opportunity? New international evidence. \emph{Economic Policy}
22(52):781--861.

\leavevmode\hypertarget{ref-Heyneman1983}{}%
2. Heyneman SP, Loxley WA (1983) The effect of primary-school quality on
academic achievement across twenty-nine high- and low-income countries.
\emph{American Journal of Sociology} 88(6):1162--1194.

\leavevmode\hypertarget{ref-Schultz2004}{}%
3. Schultz TP (2004) School subsidies for the poor: Evaluating the
Mexican Progresa poverty program. \emph{Journal of Development
Economics} 74(1):199--250.

\leavevmode\hypertarget{ref-Parker2017}{}%
4. Parker SW, Todd PE (2017) Conditional cash transfers: The case of
progresa/oportunidades. \emph{Journal of Economic Literature}
55(3):866--915.

\leavevmode\hypertarget{ref-Angelucci2013}{}%
5. Angelucci M, Attanasio O (2013) The demand for food of poor urban
Mexican households: Understanding policy impacts using structural
models. \emph{American Economic Journal: Economic Policy} 5(1):146--178.

\leavevmode\hypertarget{ref-Skoufias2005}{}%
6. Skoufias, Emmanuel (2005) \emph{PROGRESA and Its Impacts on the
Welfare}.

\leavevmode\hypertarget{ref-Galindo-Rueda2005}{}%
7. Galindo-Rueda F, Vignoles A (2005) The declining relative importance
of ability in predicting educational attainment. \emph{Journal of Human
Resources} 40(2):335--353.

\leavevmode\hypertarget{ref-Chaudhury2006}{}%
8. Chaudhury N, Hammer J, Kremer M, Muralidharan K, Rogers FH (2006)
Missing in action: Teacher and health worker absence in developing
countries. \emph{Journal of Economic Perspectives} 20(1):91--116.

\leavevmode\hypertarget{ref-Angrist2006}{}%
9. Angrist J, Bettinger E, Kremer M (2006) Long-term educational
consequences of secondary school vouchers: Evidence from administrative
records in Colombia. \emph{American Economic Review} 96(3):847--862.

\leavevmode\hypertarget{ref-Lamarche2011}{}%
10. Lamarche C (2011) Measuring the incentives to learn in Colombia
using new quantile regression approaches. \emph{Journal of Development
Economics} 96(2):278--288.

\leavevmode\hypertarget{ref-Chmielewski2019}{}%
11. Chmielewski AK (2019) The Global Increase in the Socioeconomic
Achievement Gap, 1964 to 2015. \emph{American Sociological Review}
84(3):517--544.

\leavevmode\hypertarget{ref-Golsteyn2014}{}%
12. Golsteyn BH, Schils T (2014) Gender gaps in primary school
achievement: A decomposition into endowments and returns to IQ and
non-cognitive factors. \emph{Economics of Education Review} 41:176--187.

\leavevmode\hypertarget{ref-Alon2009}{}%
13. Alon S (2009) The evolution of class inequality in higher education:
Competition, exclusion, and adaptation. \emph{American Sociological
Review} 74(5):731--755.

\leavevmode\hypertarget{ref-Marks2005}{}%
14. Marks GN (2005) Cross-national differences and accounting for social
class inequalities in education. \emph{International Sociology}
20(4):483--505.

\leavevmode\hypertarget{ref-Alon2007}{}%
15. Alon S, Tienda M (2007) Meritocracy in Higher Education.
\emph{American Sociological Review} 72:487--511.

\leavevmode\hypertarget{ref-Brown1990}{}%
16. Brown P (1990) The `Third Wave': Education and the Ideology of
Parentocracy {[}1{]}. \emph{British Journal of Sociology of Education}
11(1):65--86.

\leavevmode\hypertarget{ref-Duncan2011}{}%
17. Duncan GJ, Morris P a, Rodrigues C (2011) Does Money Really Matter?
\emph{Development Psychology} 47(5):1263--1279.

\leavevmode\hypertarget{ref-Attanasio2013}{}%
18. Attanasio O, Di Maro V, Lechene V, Phillips D (2013) Welfare
consequences of food prices increases: Evidence from rural Mexico.
\emph{Journal of Development Economics} 104:136--151.

\leavevmode\hypertarget{ref-Alexander2007}{}%
19. Alexander KL, Entwisle DR, Olson LS (2007) Lasting consequences of
the summer learning gap. \emph{American Sociological Review}
72(2):167--180.

\leavevmode\hypertarget{ref-Mare1980}{}%
20. Mare RD (1980) Social background and school continuation decisions.
\emph{Journal of the American Statistical Association} 75(370):295--305.

\leavevmode\hypertarget{ref-Reardon2016}{}%
21. Reardon SF, Portilla XA (2016) Recent Trends in Income, Racial, and
Ethnic School Readiness Gaps at Kindergarten Entry. \emph{AERA Open}
2(3):233285841665734.

\leavevmode\hypertarget{ref-Valenzuela2014}{}%
22. Valenzuela JP, Bellei C, Ríos D de los (2014) Socioeconomic school
segregation in a market-oriented educational system. The case of Chile.
\emph{Journal of Education Policy} 29(2):217--241.

\leavevmode\hypertarget{ref-Saw2016}{}%
23. Saw GK (2016) Patterns and trends in achievement gaps in Malaysian
Secondary Schools (1999--2011): gender, ethnicity, and socioeconomic
status. \emph{Educational Research for Policy and Practice}
15(1):41--54.

\leavevmode\hypertarget{ref-Downey2016}{}%
24. Downey DB, Condron DJ (2016) Fifty Years since the Coleman Report:
Rethinking the Relationship between Schools and Inequality.
\emph{Sociology of Education} 89(3):207--220.

\leavevmode\hypertarget{ref-VandeWerfhorst2018}{}%
25. Werfhorst HG van de (2018) Early tracking and socioeconomic
inequality in academic achievement: Studying reforms in nine countries.
\emph{Research in Social Stratification and Mobility} 58:22--32.

\leavevmode\hypertarget{ref-Reinke2004}{}%
26. Reinke L (2004) Globalisation and local indigenous education in
Mexico. \emph{International Review of Education} 50(5-6):483--496.

\leavevmode\hypertarget{ref-Despagne2013}{}%
27. Despagne C (2013) Indigenous Education in Mexico: Indigenous
Students' Voices. \emph{Diaspora, Indigenous, and Minority Education}
7(2):114--129.

\leavevmode\hypertarget{ref-Lopez-Gopar2007}{}%
28. López-Gopar ME (2007) Beyond the Alienating Alphabetic Literacy:
Multiliteracies in Indigenous Education in Mexico. \emph{Diaspora,
Indigenous, and Minority Education} 1(3):159--174.

\leavevmode\hypertarget{ref-Scott2002}{}%
29. Scott SL, Ip EH (2002) Empirical bayes and item-clustering effects
in a latent variable hierarchical model: A case study from the national
assessment of educational progress. \emph{Journal of the American
Statistical Association} 97(458):409--419.

\leavevmode\hypertarget{ref-Kaplan2016}{}%
30. Kaplan D (2016) Causal inference with large-scale assessments in
education from a Bayesian perspective: a review and synthesis.
\emph{Large-Scale Assessments in Education} 4(1).
doi:\href{https://doi.org/10.1186/s40536-016-0022-6}{10.1186/s40536-016-0022-6}.

\leavevmode\hypertarget{ref-h2pc}{}%
31. Gasse M, Aussem A, Elghazel H (2014) A hybrid algorithm for bayesian
network structure learning with application to multi-label learning.
\emph{Expert Systems with Applications} 41:6755--6772.



% Bibliography
% \bibliography{pnas-sample}

\end{document}

